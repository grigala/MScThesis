\chapter{Abstract}

Human face analysis is an important branch of Computer Vision. One of the interesting topics in this area of research is the analysis of a human face based on a single image. The complexity of this problem is rather high due to restricted information that is available in a single image.  Especially challenging is a task of the 3D shape reconstruction of the face based on a single image. The reconstruction process has little information to reliably recover the shape information (depth), which is the third dimension that is completely missing from an image. Using a single image does not always produce an accurate 3D reconstruction, especially in settings where lighting conditions cannot be controlled. This thesis aims to build an efficient and stable system, which adds, in addition to a photo of a face, also a 3D depth information to the reconstruction process. By incorporating a 3D depth information into this process, we aim to improve the overall quality of the final shape reconstruction, as well as, the visual appearance of the reconstructed face. We build the complete \textit{fitting} pipeline that consists of separate modules dealing with data acquisition, shape reconstruction, and analysis of color and illumination. We show that including the complete 3D depth information into the framework in the form of a point cloud indeed improves the final reconstruction quality. To obtain a 3D depth information alongside the other input requirements, we use an affordable consumer depth camera technology offered by Intel® RealSense™ platform. The pipeline is flexible enough that it can be seamlessly integrated into the existing demo-framework (the face-fitting web service\footnote{Scalismo Face Morpher — \url{https://face-morpher.scalismo.org}}) for various use-cases. The applications of this work include but not limited to photo-realistic face manipulation\footnote{Photo-realistic Face Manipulation — \url{https://gravis.dmi.unibas.ch/PMM/demo/face-manipulation/}}, 3D face modelling and analysis. 